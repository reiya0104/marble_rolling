% physics/collision-flow.tex

{%
\makeatletter

% 章タイトル
\newcommand{\sec@collision@flow}[1]{%
  \stringcases
    {#1}%
    {%
      {1}{ビー玉の位置をタイルの座標系に変換}%
      {2}{ビー玉の各軸の最大最小の座標に基づくタイルの位置を抽出}%
      {3}{抽出された8つのタイルの位置とその間のタイルの位置についてタイルが存在するものを取り出す}%
      {4}{存在するタイルについてそのタイルとビー玉との距離を計算}%
      {5}{距離がビー玉の半径よりも小さかったら衝突してると判定}%
    }%
    {[else]}%
}

\newcommand{\e}{\bm{e}}

\newcommand{\e@x}{\e_x}
\newcommand{\e@y}{\e_y}
\newcommand{\e@z}{\e_z}

\newcommand{\e@tile}{\e^\prime}
\newcommand{\e@tile@x}{\e@tile_x}
\newcommand{\e@tile@y}{\e@tile_y}
\newcommand{\e@tile@z}{\e@tile_z}

\newcommand{\O@tile}{{\rm O}^\prime}
\newcommand{\x@tile}{x^\prime}
\newcommand{\y@tile}{y^\prime}
\newcommand{\z@tile}{z^\prime}

\newcommand{\v@O@tile}{\bm{v}_{\O@tile}}

\newcommand{\v@P}{\bm{v}_{\rm P}}
\newcommand{\v@tile@P}{\bm{v}^\prime_{\rm P}}
\subsection{衝突判定}

\begin{center}
  % physics/marble_sphere.tex

% 球を描画

{%
% \usetikzlibrary{intersections,calc,arrows.meta}

% 参考
% https://tex.stackexchange.com/questions/348609/draw-a-3d-sphere-with-radius-with-tikz
\begin{tikzpicture}
  % 面
  \coordinate (O) at (0,0); % 原点 O
  \coordinate (L) at (-2,0); % 左
  \coordinate (R) at (2,0); % 右
  \coordinate (U) at (0,2); % 上
  \coordinate (D) at (0,-2); % 下
  \coordinate (vB) at (1,1); % 後ろ方向
  \coordinate (vF) at (-1,-1); % 前方向

  \draw[dashed, name path=arcB] (R) arc (0:180:2 and 0.6); % 後ろ曲線
  \draw[black!0, name path=arcF] (L) arc (180:360:2 and 0.6); % 前曲線 (後ほど描画)

  % Bを求める
  \draw[black!0, name path=lineB] (O) -- (vB);
  \path[name intersections={of= arcB and lineB, by={B}}];
  
  % Fを求める
  \draw[black!0, name path=lineF] (O) -- (vF);
  \path[name intersections={of= arcF and lineF, by={F}}];

  % 頂点
  \coordinate (C1) at ($(L)+(B)-(O)+(U)$); % 上左後ろ
  \coordinate (C2) at ($(L)+(F)-(O)+(U)$); % 上左前
  \coordinate (C3) at ($(R)+(F)-(O)+(U)$); % 上右前
  \coordinate (C4) at ($(R)+(B)-(O)+(U)$); % 上右後ろ
  \coordinate (C5) at ($(L)+(B)-(O)+(D)$); % 下左後ろ
  \coordinate (C6) at ($(L)+(F)-(O)+(D)$); % 下左前
  \coordinate (C7) at ($(R)+(F)-(O)+(D)$); % 下右前
  \coordinate (C8) at ($(R)+(B)-(O)+(D)$); % 下右後ろ

  % 辺
  \coordinate (E1) at ($(L)+(B)-(O)$); % 左後ろ
  \coordinate (E2) at ($(L)+(F)-(O)$); % 左前
  \coordinate (E3) at ($(R)+(F)-(O)$); % 右前
  \coordinate (E4) at ($(R)+(B)-(O)$); % 右後ろ

  \coordinate (S1) at ($(L)+(U)$); % 
  \coordinate (S2) at ($(R)+(U)$); % 
  \coordinate (S3) at ($(R)+(D)$); % 
  \coordinate (S4) at ($(L)+(D)$); % 

  \coordinate (M1) at ($(B)+(U)$); % 
  \coordinate (M2) at ($(F)+(U)$); % 
  \coordinate (M3) at ($(F)+(D)$); % 
  \coordinate (M4) at ($(B)+(D)$); % 


  % 線の描画
  \draw[dashed] (O) -- (R);
  \draw[dashed] (O) -- (L);
  \draw[dashed] (O) -- (U);
  \draw[dashed] (O) -- (D);
  \draw[dashed] (O) -- (B);
  \draw[dashed] (O) -- (F);

  \draw[ultra thin] (C1) -- (C2);
  % \draw[ultra thin] (C2) -- (C3); % (後ほど描画)
  \draw[ultra thin] (C3) -- (C4);
  \draw[ultra thin] (C4) -- (C1);

  \draw[ultra thin] (C5) -- (C6) -- (C7) -- (C8) -- cycle;

  \draw[ultra thin] (C1) -- (C5);
  \draw[ultra thin] (C2) -- (C6);
  % \draw[ultra thin] (C3) -- (C7); % (後ほど描画)
  \draw[ultra thin] (C4) -- (C8);
  

  % 点の描画
  \fill (O) circle (0.06) node[above left]{O}; % 点O
  \fill (L) circle (0.06) node[left]{L}; % 点L
  \fill (R) circle (0.06) node[right]{R}; % 点R
  \fill (U) circle (0.06) node[above]{U}; % 点U
  \fill (D) circle (0.06) node[below]{D}; % 点D
  \fill (B) circle (0.06) node[above right]{B}; % 点B

  \fill (C1) circle (0.06) node[above left]{C$_1$}; % 点C1
  \fill (C2) circle (0.06) node[left]{C$_2$}; % 点C2
  % \fill (C3) circle (0.06) node[right]{C$_3$}; % 点C3 (後ほど描画)
  \fill (C4) circle (0.06) node[above right]{C$_4$}; % 点C4

  \fill (C5) circle (0.06) node[above left]{C$_5$}; % 点C5
  \fill (C6) circle (0.06) node[below left]{C$_6$}; % 点C6
  \fill (C7) circle (0.06) node[below right]{C$_7$}; % 点C7
  \fill (C8) circle (0.06) node[above right]{C$_8$}; % 点C8

  \fill (E1) circle (0.06) node[above left]{E$_1$}; % 点E1
  \fill (E2) circle (0.06) node[below left]{E$_2$}; % 点E2
  % \fill (E3) circle (0.06) node[below right]{E$_3$}; % 点E3 (後ほど描画)
  \fill (E4) circle (0.06) node[above right]{E$_4$}; % 点E4

  \fill (S1) circle (0.06) node[left]{S$_1$}; % 点S1
  \fill (S2) circle (0.06) node[right]{S$_2$}; % 点S2
  \fill (S3) circle (0.06) node[right]{S$_3$}; % 点S3
  \fill (S4) circle (0.06) node[left]{S$_4$}; % 点S4

  \fill (M1) circle (0.06) node[above]{M$_1$}; % 点M1
  % \fill (M2) circle (0.06) node[above]{M$_2$}; % 点M2 (後ほど描画)
  \fill (M3) circle (0.06) node[below]{M$_3$}; % 点M3
  \fill (M4) circle (0.06) node[above right]{M$_4$}; % 点M4

  % 球・影の描画
  \shade[ball color = gray!10, opacity = 0.5] (O) circle (2cm);
  \draw (O) circle (2cm);

  % 影の上から描画するものたち

  \draw (L) arc (180:360:2 and 0.6); % 前曲線 (描画)

  \draw[ultra thin] (C2) -- (C3);
  \draw[ultra thin] (C3) -- (C7);

  \fill (F) circle (0.06) node[below left]{F}; % 点F
  \fill (C3) circle (0.06) node[right]{C$_3$}; % 点C3
  \fill (E3) circle (0.06) node[below right]{E$_3$}; % 点E3
  \fill (M2) circle (0.06) node[above]{M$_2$}; % 点M2
\end{tikzpicture}

}


\end{center}

ビー玉の衝突判定の流れ

\begin{enumerate}
  \item \sec@collision@flow{1} (\ref{sec:collision-flow-1})
  \item \sec@collision@flow{2} (\ref{sec:collision-flow-2})
  \item \sec@collision@flow{3} (\ref{sec:collision-flow-3})
  \item \sec@collision@flow{4} (\ref{sec:collision-flow-4})
  \item \sec@collision@flow{5} (\ref{sec:collision-flow-5})
\end{enumerate}

それぞれについて見ていく.

% physics/collision-flow/1.tex

{%
\makeatletter

% ビー玉の位置をタイルの座標系に変換
\subsubsection{\sec@collision@flow{1}}
\label{sec:collision-flow-1}

タイルの(直交)座標系の基本ベクトルを \(\e@tile@x, \e@tile@y, \e@tile@z\)
とおく.
また,タイルの座標系の原点 \(\O@tile\) のワールドの座標系における位置ベクトルを
\(\v@O@tile\) とする.

ここで,ビー玉について,ビー玉のある点をPとし,
ワールドの座標系における位置ベクトルを \(\v@P\),
タイルの座標系における位置ベクトルを \(\v@tile@P\)
とする.

このとき,\(\v@tile@P\) は \(\e@tile@x, \e@tile@y, \e@tile@z, \v@O@tile, \v@P\) を用いて
\begin{align}
  \v@tile@P =
  \begin{pmatrix}
    \e@tile@x & \e@tile@y & \e@tile@z
  \end{pmatrix}^{-1}
  (\v@P - \v@O@tile)
\end{align}
と表せる.

% physics/collision-flow/1_tikzpicture.tex
{%

% xyz軸の長さ
\newcommand{\@size@x}{7}
\newcommand{\@size@y}{5}
\newcommand{\@size@z}{3}

% 点P の座標
\newcommand{\@x}{3}
\newcommand{\@y}{4}
\newcommand{\@z}{2}

% タイルの座標の原点 \O@tile の座標
\newcommand{\O@tile@x}{6.5}
\newcommand{\O@tile@y}{1}
\newcommand{\O@tile@z}{2.5}

% \e@tile@xの座標
\newcommand{\e@tile@x@x}{1/3}
\newcommand{\e@tile@x@y}{1/2}
\newcommand{\e@tile@x@z}{0}
% \e@tile@yの座標
\newcommand{\e@tile@y@x}{-1/4}
\newcommand{\e@tile@y@y}{1/2}
\newcommand{\e@tile@y@z}{-1/3}
% \e@tile@zの座標
\newcommand{\e@tile@z@x}{-3/4}
\newcommand{\e@tile@z@y}{0}
\newcommand{\e@tile@z@z}{1/3}

\begin{center}
  \begin{tikzpicture}
    % 点
    % ワールド座標系
    \coordinate (O) at (0,0); % 原点 O
    \coordinate (e_x) at (1,0);
    \coordinate (e_y) at (0,1);
    \coordinate (e_z) at ($5/6*(-1/2,{-sqrt(3)/2})$);
    \coordinate (P) at ($\@x*(e_x)+\@y*(e_y)+\@z*(e_z)$);

    % タイルの座標系
    \coordinate (O_tile) at ($\O@tile@x*(e_x)+\O@tile@y*(e_y)+\O@tile@z*(e_z)$);
    \coordinate (e_tile_x) at ($\e@tile@x@x*(e_x)+\e@tile@x@y*(e_y)+\e@tile@x@z*(e_z)$);
    \coordinate (e_tile_y) at ($\e@tile@y@x*(e_x)+\e@tile@y@y*(e_y)+\e@tile@y@z*(e_z)$);
    \coordinate (e_tile_z) at ($\e@tile@z@x*(e_x)+\e@tile@z@y*(e_y)+\e@tile@z@z*(e_z)$);
    
    % \v@tile@P の計算
    \tikzmath{%
      % 3*3行列の逆行列を求める
      % 行列に代入
      %
      % [\m@a@a \m@a@b \m@a@c]
      % [\m@b@a \m@b@b \m@b@c]
      % [\m@c@a \m@c@b \m@c@c]
      %
      \m@a@a = \e@tile@x@x;
      \m@a@b = \e@tile@y@x;
      \m@a@c = \e@tile@z@x;
      \m@b@a = \e@tile@x@y;
      \m@b@b = \e@tile@y@y;
      \m@b@c = \e@tile@z@y;
      \m@c@a = \e@tile@x@z;
      \m@c@b = \e@tile@y@z;
      \m@c@c = \e@tile@z@z;
      %
      % 行列式
      \@det = \m@a@a * \m@b@b * \m@c@c
        + \m@a@b * \m@b@c * \m@c@a
        + \m@a@c * \m@b@a * \m@c@b
        - \m@a@a * \m@b@c * \m@c@b
        - \m@a@b * \m@b@a * \m@c@c
        - \m@a@c * \m@b@b * \m@c@a;
      %
      % 逆行列
      \i@a@a =  (\m@b@b * \m@c@c - \m@b@c * \m@c@b) / \@det;
      \i@a@b = -(\m@a@b * \m@c@c - \m@a@c * \m@c@b) / \@det;
      \i@a@c =  (\m@a@b * \m@b@c - \m@a@c * \m@b@b) / \@det;
      \i@b@a = -(\m@b@a * \m@c@c - \m@b@c * \m@c@a) / \@det;
      \i@b@b =  (\m@a@a * \m@c@c - \m@a@c * \m@c@a) / \@det;
      \i@b@c = -(\m@a@a * \m@b@c - \m@a@c * \m@b@a) / \@det;
      \i@c@a =  (\m@b@a * \m@c@b - \m@b@b * \m@c@a) / \@det;
      \i@c@b = -(\m@a@a * \m@c@b - \m@a@b * \m@c@a) / \@det;
      \i@c@c =  (\m@a@a * \m@b@b - \m@a@b * \m@b@a) / \@det;
      %
      % ベクトルに代入
      \v@x = \@x - \O@tile@x;
      \v@y = \@y - \O@tile@y;
      \v@z = \@z - \O@tile@z;
      %
      % 逆行列 * ベクトル
      \iv@x = \i@a@a * \v@x + \i@a@b * \v@y + \i@a@c * \v@z;
      \iv@y = \i@b@a * \v@x + \i@b@b * \v@y + \i@b@c * \v@z;
      \iv@z = \i@c@a * \v@x + \i@c@b * \v@y + \i@c@c * \v@z;
      %
      % 数値を代入
      \v@tile@P@x = \iv@x;
      \v@tile@P@y = \iv@y;
      \v@tile@P@z = \iv@z;
    }
    \coordinate (v_tile_P) at ($\v@tile@P@x*(e_tile_x)+\v@tile@P@y*(e_tile_y)+\v@tile@P@z*(e_tile_z)$);

    % 点の描画
    \fill (O) circle (0.06) node[above left]{O}; % 点O
    % xyz軸の描画
    \draw[->,>=stealth,semithick] (O)--($\@size@x*(e_x)$)node[above]{$x$}; % x軸
    \draw[->,>=stealth,semithick] (O)--($\@size@y*(e_y)$)node[left]{$y$}; % y軸
    \draw[->,>=stealth,semithick] (O)--($\@size@z*(e_z)$)node[above left]{$z$}; % z軸

    % 基本ベクトルの描画
    % \draw[->,>=stealth,very thick] (O)--(e_x)node[above]{$\e@x$}; % e_x
    % \draw[->,>=stealth,very thick] (O)--(e_y)node[left]{$\e@y$}; % e_y
    % \draw[->,>=stealth,very thick] (O)--(e_z)node[above left]{$\e@z$}; % e_z

    % \v@P の描画
    \draw[->,>=stealth,very thick] (O)--(P)node[xshift=-20,yshift=-40]{$\v@P$}; % v_P

    \fill (P) circle (0.06) node[above left]{P}; % 点P
    % 点Pのx座標からの点線
    \draw[dashed, thin] ($\@x*(e_x)$)--($\@x*(e_x)+\@y*(e_y)$);
    \draw[dashed, thin] ($\@x*(e_x)$)--($\@x*(e_x)+\@z*(e_z)$);
    % 点Pのy座標からの点線
    \draw[dashed, thin] ($\@y*(e_y)$)--($\@y*(e_y)+\@x*(e_x)$);
    \draw[dashed, thin] ($\@y*(e_y)$)--($\@y*(e_y)+\@z*(e_z)$);
    % 点Pのz座標からの点線
    \draw[dashed, thin] ($\@z*(e_z)$)--($\@z*(e_z)+\@y*(e_y)$);
    \draw[dashed, thin] ($\@z*(e_z)$)--($\@z*(e_z)+\@x*(e_x)$);
    % 点Pへの点線
    \draw[dashed, thin] ($\@x*(e_x)+\@y*(e_y)$)--(P);
    \draw[dashed, thin] ($\@y*(e_y)+\@z*(e_z)$)--(P);
    \draw[dashed, thin] ($\@z*(e_z)+\@x*(e_x)$)--(P);


    \fill (O_tile) circle (0.06) node[right]{$\O@tile$}; % 点 \O@tile の描画
    \draw[->,>=stealth,very thick] (O)--(O_tile)node[left,xshift=-40,yshift=15]{$\v@O@tile$}; % v_P

    \draw[dashed, thin] (O)--($\O@tile@x*(e_x)$); % 点 \O@tile の描画
    % 点 \O@tile のx座標からの点線
    \draw[dashed, thin] ($\O@tile@x*(e_x)$)--($\O@tile@x*(e_x)+\O@tile@y*(e_y)$);
    \draw[dashed, thin] ($\O@tile@x*(e_x)$)--($\O@tile@x*(e_x)+\O@tile@z*(e_z)$);
    % 点 \O@tile のy座標からの点線
    \draw[dashed, thin] ($\O@tile@y*(e_y)$)--($\O@tile@y*(e_y)+\O@tile@x*(e_x)$);
    \draw[dashed, thin] ($\O@tile@y*(e_y)$)--($\O@tile@y*(e_y)+\O@tile@z*(e_z)$);
    % 点 \O@tile のz座標からの点線
    \draw[dashed, thin] ($\O@tile@z*(e_z)$)--($\O@tile@z*(e_z)+\O@tile@y*(e_y)$);
    \draw[dashed, thin] ($\O@tile@z*(e_z)$)--($\O@tile@z*(e_z)+\O@tile@x*(e_x)$);
    % 点Pへの点線
    \draw[dashed, thin] ($\O@tile@x*(e_x)+\O@tile@y*(e_y)$)--(O_tile);
    \draw[dashed, thin] ($\O@tile@y*(e_y)+\O@tile@z*(e_z)$)--(O_tile);
    \draw[dashed, thin] ($\O@tile@z*(e_z)+\O@tile@x*(e_x)$)--(O_tile);
    
    % タイルの座標系の基本ベクトル
    \draw[->,>=stealth,very thick, blue] (O_tile)--($(O_tile)+(e_tile_x)$)node[right]{$\e@tile@x$}; % e_tile_x
    \draw[->,>=stealth,very thick, blue] (O_tile)--($(O_tile)+(e_tile_y)$)node[above left]{$\e@tile@y$}; % e_tile_y
    \draw[->,>=stealth,very thick, blue] (O_tile)--($(O_tile)+(e_tile_z)$)node[below]{$\e@tile@z$}; % e_tile_z

    % v_tile_P の描画
    % \draw[->,>=stealth,very thick] (O_tile)--(P)node[left,xshift=-20,yshift=-7]{$\v@tile@P$}; % v_tile_P
    \draw[->,>=stealth,very thick, blue](O_tile)--($(O_tile)+(v_tile_P)$)
      node[right,xshift=23,yshift=-20]{$\v@tile@P$}; % v_tile_P
    
    % O_tile から座標への点線
    \draw[dashed, thin, blue] (O_tile)--($(O_tile)+\v@tile@P@x*(e_tile_x)$);
    \draw[dashed, thin, blue] (O_tile)--($(O_tile)+\v@tile@P@y*(e_tile_y)$);
    \draw[dashed, thin, blue] (O_tile)--($(O_tile)+\v@tile@P@z*(e_tile_z)$);
    % タイルの座標系での点Pのx座標からの点線
    \draw[dashed, thin, blue] ($(O_tile)+\v@tile@P@x*(e_tile_x)$)--($(O_tile)+\v@tile@P@x*(e_tile_x)+\v@tile@P@y*(e_tile_y)$);
    \draw[dashed, thin, blue] ($(O_tile)+\v@tile@P@x*(e_tile_x)$)--($(O_tile)+\v@tile@P@x*(e_tile_x)+\v@tile@P@z*(e_tile_z)$);
    % タイルの座標系での点Pのy座標からの点線
    \draw[dashed, thin, blue] ($(O_tile)+\v@tile@P@y*(e_tile_y)$)--($(O_tile)+\v@tile@P@y*(e_tile_y)+\v@tile@P@x*(e_tile_x)$);
    \draw[dashed, thin, blue] ($(O_tile)+\v@tile@P@y*(e_tile_y)$)--($(O_tile)+\v@tile@P@y*(e_tile_y)+\v@tile@P@z*(e_tile_z)$);
    % タイルの座標系での点Pのz座標からの点線
    \draw[dashed, thin, blue] ($(O_tile)+\v@tile@P@z*(e_tile_z)$)--($(O_tile)+\v@tile@P@z*(e_tile_z)+\v@tile@P@y*(e_tile_y)$);
    \draw[dashed, thin, blue] ($(O_tile)+\v@tile@P@z*(e_tile_z)$)--($(O_tile)+\v@tile@P@z*(e_tile_z)+\v@tile@P@x*(e_tile_x)$);
    % 点Pへの点線
    \draw[dashed, thin, blue] ($(O_tile)+\v@tile@P@x*(e_tile_x)+\v@tile@P@y*(e_tile_y)$)--(P);
    \draw[dashed, thin, blue] ($(O_tile)+\v@tile@P@y*(e_tile_y)+\v@tile@P@z*(e_tile_z)$)--(P);
    \draw[dashed, thin, blue] ($(O_tile)+\v@tile@P@z*(e_tile_z)+\v@tile@P@x*(e_tile_x)$)--(P);
  \end{tikzpicture}
\end{center}
}


}

% physics/collision-flow/2.tex
{%
% ビー玉の各軸の最大最小の座標に基づくタイルの位置を抽出
\subsubsection{\sec@collision@flow{2}}
\label{sec:collision-flow-2}

以下はタイルの座標系として考える.

本題に入る前に,タイルの構成と,点とタイルの位置関係について述べる.

\noindent
{\bf タイルの構成}

タイルは直方体でできており,下の図のように敷き詰められている.
ここで, \(s = \verb|TILE_SHORT_WIDTH|, \,
l = \verb|TILE_LONG_WIDTH|\) とする.

% physics/collision-flow/2_tikzpicture.tex
{%

% xyz軸の長さ
\newcommand{\@size@x}{7.5}
\newcommand{\@size@y}{3}
\newcommand{\@size@z}{3.5}

\newcommand{\s@length}{0.8}
\newcommand{\l@length}{2.0}

\newcommand{\max@i}{4}
\newcommand{\max@j}{1}
\newcommand{\max@k}{1}

\begin{center}
  \begin{tikzpicture}
    % 点
    % ワールド座標系
    \coordinate (O_tile) at (0,0);
    \coordinate (e_tile_x) at (1,0);
    \coordinate (e_tile_y) at (0,1);
    \coordinate (e_tile_z) at ($5/6*(-1/2,{-sqrt(3)/2})$);
    \coordinate (O_tile_draw) at ($(O_tile)+(0, -\s@length)$);

    % 座標系の描画
    \fill (O_tile) circle (0.06) node[below right]{$\O@tile$};
    \draw[->,>=stealth,very thick] (O_tile)--($\@size@x*(e_tile_x)$)node[above]{$\x@tile$}; % x座標
    \draw[->,>=stealth,very thick] (O_tile)--($\@size@y*(e_tile_y)$)node[left]{$\y@tile$}; % y座標
    \draw[->,>=stealth,very thick] (O_tile)--($\@size@z*(e_tile_z)$)node[below]{$\z@tile$}; % z座標
    
    \tikzmath{%
      \max@x@cood = div(\max@i+2,2)*\s@length+div(\max@i+1,2)*\l@length;
      \max@y@cood = div(\max@j+2,2)*\s@length+div(\max@j+1,2)*\l@length;
      \max@z@cood = div(\max@k+2,2)*\s@length+div(\max@k+1,2)*\l@length;
    }
    \draw[dotted] ($(O_tile)+(0, -\s@length)$)--($(O_tile)+(0, -\s@length)+\@size@x*(e_tile_x)$);
    \draw[dotted] ($(O_tile)+(0, -\s@length)$)--($(O_tile)+(0, -\s@length)+\@size@y*(e_tile_y)$);
    \draw[dotted] ($(O_tile)+(0, -\s@length)$)--($(O_tile)+(0, -\s@length)+\@size@z*(e_tile_z)$);
    % 繰り返し
    \foreach \i in {0, ..., \max@i}
    {%
      \foreach \j in {0, ..., \max@j}
      {%
        \foreach \k in {0, ..., \max@k}
        {%
          % タイルの枠組みの描画 
          \tikzmath{%
            \tile@x = div(\i+1,2)*\s@length+div(\i,2)*\l@length;
            \tile@y = div(\j+1,2)*\s@length+div(\j,2)*\l@length;
            \tile@z = div(\k+1,2)*\s@length+div(\k,2)*\l@length;
            %
            \tile@next@x = div(\i+2,2)*\s@length+div(\i+1,2)*\l@length;
            \tile@next@y = div(\j+2,2)*\s@length+div(\j+1,2)*\l@length;
            \tile@next@z = div(\k+2,2)*\s@length+div(\k+1,2)*\l@length;
            %
            % 表示する用の値
            \tile@node@x@s@count = div(\i+2,2);
            \tile@node@x@l@count = div(\i+1,2);
            \tile@node@y@s@count = div(\j,2);
            \tile@node@y@l@count = div(\j+1,2);
            \tile@node@z@s@count = div(\k+2,2);
            \tile@node@z@l@count = div(\k+1,2);
          }

          % タイルの影の描画
          \ifthenelse{\(\j=0 \AND \k=1\) \OR \(\j=0 \AND \i=3\)}{
            \shade ($(O_tile_draw)+\tile@next@x*(e_tile_x)+\tile@next@y*(e_tile_y)+\tile@z*(e_tile_z)$)
              --($(O_tile_draw)+\tile@next@x*(e_tile_x)+\tile@next@y*(e_tile_y)+\tile@next@z*(e_tile_z)$)
              --($(O_tile_draw)+\tile@next@x*(e_tile_x)+\tile@y*(e_tile_y)+\tile@next@z*(e_tile_z)$)
              --($(O_tile_draw)+\tile@next@x*(e_tile_x)+\tile@y*(e_tile_y)+\tile@z*(e_tile_z)$)
              --cycle;
            \shade ($(O_tile_draw)+\tile@next@x*(e_tile_x)+\tile@next@y*(e_tile_y)+\tile@z*(e_tile_z)$)
              --($(O_tile_draw)+\tile@next@x*(e_tile_x)+\tile@next@y*(e_tile_y)+\tile@next@z*(e_tile_z)$)
              --($(O_tile_draw)+\tile@x*(e_tile_x)+\tile@next@y*(e_tile_y)+\tile@next@z*(e_tile_z)$)
              --($(O_tile_draw)+\tile@x*(e_tile_x)+\tile@next@y*(e_tile_y)+\tile@z*(e_tile_z)$)
              --cycle;
            \shade ($(O_tile_draw)+\tile@x*(e_tile_x)+\tile@next@y*(e_tile_y)+\tile@next@z*(e_tile_z)$)
              --($(O_tile_draw)+\tile@next@x*(e_tile_x)+\tile@next@y*(e_tile_y)+\tile@next@z*(e_tile_z)$)
              --($(O_tile_draw)+\tile@next@x*(e_tile_x)+\tile@y*(e_tile_y)+\tile@next@z*(e_tile_z)$)
              --($(O_tile_draw)+\tile@x*(e_tile_x)+\tile@y*(e_tile_y)+\tile@next@z*(e_tile_z)$)
              --cycle;
          }{};

          \ifthenelse{\j=0 \AND \k=0}{
            \draw[dotted] ($(O_tile_draw)+\tile@next@x*(e_tile_x)+\@size@y*(e_tile_y)$)
              --($(O_tile_draw)+\tile@next@x*(e_tile_x)$)
              --($(O_tile_draw)+\tile@next@x*(e_tile_x)+\@size@z*(e_tile_z)$);
              \fill($(O_tile)+\tile@next@x*(e_tile_x)$) circle (0.06) node[%
              \ifodd\i below\else above\fi
              ]{$\tile@node@x@s@count \, s + \tile@node@x@l@count \, l$};
          }{};
          \ifthenelse{\k=0 \AND \i=0}{
            \draw[dotted] ($(O_tile_draw)+\tile@next@y*(e_tile_y)+\@size@z*(e_tile_z)$)
              --($(O_tile_draw)+\tile@next@y*(e_tile_y)$)
              --($(O_tile_draw)+\tile@next@y*(e_tile_y)+\@size@x*(e_tile_x)$);
            \fill($(O_tile_draw)+\tile@next@y*(e_tile_y)$) circle (0.06) node[left]{$\tile@node@y@s@count \, s + \tile@node@y@l@count \, l$};
          }{};
          \ifthenelse{\i=0 \AND \j=0}{
            \draw[dotted] ($(O_tile_draw)+\tile@next@z*(e_tile_z)+\@size@x*(e_tile_x)$)
              --($(O_tile_draw)+\tile@next@z*(e_tile_z)$)
              --($(O_tile_draw)+\tile@next@z*(e_tile_z)+\@size@y*(e_tile_y)$);
            \fill($(O_tile)+\tile@next@z*(e_tile_z)$) circle (0.06) node[left]{$\tile@node@z@s@count \, s + \tile@node@z@l@count \, l$};
          }{};
          \draw[dotted] ($(O_tile_draw)+\tile@next@x*(e_tile_x)+\tile@next@y*(e_tile_y)+\tile@z*(e_tile_z)$)
            --($(O_tile_draw)+\tile@next@x*(e_tile_x)+\tile@next@y*(e_tile_y)+\tile@next@z*(e_tile_z)$)
            --($(O_tile_draw)+\tile@next@x*(e_tile_x)+\tile@y*(e_tile_y)+\tile@next@z*(e_tile_z)$);
          
          \draw[dotted] ($(O_tile_draw)+\tile@x*(e_tile_x)+\tile@next@y*(e_tile_y)+\tile@next@z*(e_tile_z)$)
            --($(O_tile_draw)+\tile@next@x*(e_tile_x)+\tile@next@y*(e_tile_y)+\tile@next@z*(e_tile_z)$);
          
        }
      }
    }
  \end{tikzpicture}
\end{center}
}


この図のように,
\(\x@tile, \y@tile, \z@tile\) 軸のそれぞれの方向における \(i, j, k\) 番目(\(i, j, k = 0, 1, 2, \dots\))のタイルは,

\begin{align}
  \label{eq:tile-aria}
  \begin{alignedat}{9}
    &\x@tile \, \text{軸:} &\quad & 
    &\biggl\lfloor \frac{i + 1}{2} \biggr\rfloor \, & s + & \biggl\lfloor \frac{i}{2} \biggr\rfloor \, & l
    && \le \x@tile
    < & \biggl\lfloor \frac{i + 2}{2} \biggr\rfloor \, & s + & \biggl\lfloor \frac{i + 1}{2} \biggr\rfloor \, & l \\
    &\y@tile \, \text{軸:} & & 
    &\biggl\lfloor \frac{j - 1}{2} \biggr\rfloor \, & s + & \biggl\lfloor \frac{j}{2} \biggr\rfloor \, & l
    && \le \y@tile
    < & \biggl\lfloor \frac{j}{2} \biggr\rfloor \, & s + & \biggl\lfloor \frac{j + 1}{2} \biggr\rfloor \, & l \\
    &\z@tile \, \text{軸:} & & 
    &\biggl\lfloor \frac{k + 1}{2} \biggr\rfloor \, & s + & \biggl\lfloor \frac{k}{2} \biggr\rfloor \, & l
    && \le \z@tile
    < & \biggl\lfloor \frac{k + 2}{2} \biggr\rfloor \, & s + & \biggl\lfloor \frac{k + 1}{2} \biggr\rfloor \, & l
  \end{alignedat}
\end{align}

の領域に配置される.
ここで,
\(\lfloor x \rfloor\) は \(n \le x < n + 1\) となる整数 \(n\) のことである.
今後,この領域にあるタイルのことを \([i, j, k]\) と表すとする.

例えば, \([1, 2, 3]\) は,
\(( s \le \x@tile < s + l \ \text{かつ} \ s \le \y@tile < s + l \ \text{かつ} \ 2 \, s + l \le \z@tile < 2 \, s + 2 \, l )\)
という領域のタイルを表す.

\noindent
{\bf 点とタイルの位置関係}

このようなタイルの構成について,ある点P \((a, b, c)\) がどのタイル \([i, j, k]\) に含まれているのかを考えたい.
これは,上の 式\eqref{eq:tile-aria} から逆算すると,
\(i, j, k\) はそれぞれ

\begin{align}
  \begin{aligned}
    i &= 2 \biggl\lfloor \frac{a}{s + l} \biggr\rfloor
      + \biggl\lfloor \frac{a - s}{s + l} - \biggl\lfloor \frac{a}{s + l} \biggr\rfloor \biggr\rfloor
      + 1 \eqqcolon f_i(a) \\
    j &= 2 \biggl\lfloor \frac{b + s}{s + l} \biggr\rfloor
      + \biggl\lfloor \frac{b}{s + l} - \biggl\lfloor \frac{b + s}{s + l} \biggr\rfloor \biggr\rfloor
      + 1 \eqqcolon f_j(b) \\
    k &= 2 \biggl\lfloor \frac{c}{s + l} \biggr\rfloor
      + \biggl\lfloor \frac{c - s}{s + l} - \biggl\lfloor \frac{c}{s + l} \biggr\rfloor \biggr\rfloor
      + 1 \eqqcolon f_k(c)
  \end{aligned}
\end{align}
となる.
このように,点P \((a, b, c)\) からその点を含むタイル \([i, j, k]\)
を返す関数を
\begin{align}
  \label{eq:get-coordinates}
  f((a, b, c)) = [f_i(a), f_j(b), f_k(c)]
\end{align}
とする.

求め方:
\begin{quotation}
  \(i\) を求める.\(i\) は0以上の整数であるから,
  ある非負整数 \(n\) を用いて
  \(i = 2 n \, \text{または} \, i = 2 (n + 1)\) と表せる.
  このとき,式\eqref{eq:tile-aria} の \(\x@tile\) の範囲はそれぞれ
  \begin{alignat*}{3}
    &i = 2 n \text{のとき}: & n (s + l) &\le \x@tile < n (s + l) + s \\
    &i = 2 n + 1 \text{のとき}: \quad & n (s + l) + s &\le \x@tile < (n + 1) (s + l)
  \end{alignat*}
  となる.各辺を \(s + l\) で割ると,\(s + l > 0\) から,
  \begin{alignat*}{3}
    &i = 2 n \text{のとき}: & n &\le \frac{\x@tile}{s + l} < n + \frac{s}{s + l} \\
    &i = 2 n + 1 \text{のとき}: \quad & n + \frac{s}{s + l} &\le \frac{\x@tile}{s + l} < n + 1
  \end{alignat*}
  がいえる.これより,\(a\) について
  \(n^\prime \le \frac{a}{s + l} < n^\prime + \frac{s}{s + l}\)
  を満たすような非負整数 \(n^\prime\) が存在すれば,
  この \(n^\prime\) について \(i = 2 n^\prime\) が成り立ち,
  \(n^\prime + \frac{s}{s + l} \le \frac{a}{s + l} < n^\prime + 1\)
  を満たすような非負整数 \(n^\prime\) が存在すれば,
  この \(n^\prime\) について \(i = 2 n^\prime + 1\) が成り立つ.
  任意の \(a\) と \(n^\prime\) について \(\frac{a}{s + l} < n^\prime + \frac{s}{s + l} \le \frac{a}{s + l}\)
  は成立しないので,
  上の二つはどちらかしか成り立たない.

  \(m \coloneqq \bigl\lfloor \frac{a}{s + l} \bigr\rfloor\)
  とすると,\( m \le \frac{a}{s + l} < m + 1\) より,
  \(n^\prime = m\) である.
  このとき,
  \begin{gather*}
    i = 2 m + g(a)\\
    \left(
    \text{ここで,} \, g(a) = \left\{\,\begin{alignedat}{3}
      &0 \quad \Bigl( (m \le ) &&\frac{a}{s + l} < m + \frac{s}{s + l} \, \text{のとき}\Bigr)\\
      &1 \quad \Bigl( (m + 1 > ) &&\frac{a}{s + l} \ge m + \frac{s}{s + l} \, \text{のとき}\Bigr)
    \end{alignedat}
    \right.
    \right)
  \end{gather*}
  が成り立つ.
  \(g(a)\) は複数の表示があるが,ここではガウス記号(床関数)を用いると\\
  \(g(a) = \bigl\lfloor \frac{a}{s + l} - \bigl(m + \frac{s}{s + l} \bigr)\bigr\rfloor + 1
  = \bigl\lfloor \frac{a - s}{s + l} - m \bigr\rfloor + 1\)
  と表せる.
  よって,
  \(i = 2 \bigl\lfloor \frac{a}{s + l} \bigr\rfloor
  + \bigl\lfloor \frac{a - s}{s + l} - \bigl\lfloor \frac{a}{s + l} \bigr\rfloor \bigr\rfloor
  + 1\)
  となる.\(j, k\) についても同様にして求められる.
\end{quotation}

\vskip\baselineskip

ここから本題に入る.
ビー玉の半径を \(r\) とし,
ビー玉の中心がタイルの座標系において \((a, b, c)\)
の位置にあるとする.

このとき,各軸の最大最小の座標に基づくタイルの位置を求める.
\(\x@tile\) 軸における最大・最小値はそれぞれ
\(a + r, a - r\) である.
同様に,\(\y@tile\) 軸では \(b + r, b - r\),
\(\z@tile\) 軸では \(c + r, c - r\) である.
これより,求めるタイルは
\begin{gather*}
  T_1 \coloneqq f((a - r, b - r, c - r)), \quad
  T_2 \coloneqq f((a + r, b - r, c - r)), \\
  T_3 \coloneqq f((a - r, b + r, c - r)), \quad
  T_4 \coloneqq f((a + r, b + r, c - r)), \\
  T_5 \coloneqq f((a - r, b - r, c + r)), \quad
  T_6 \coloneqq f((a + r, b - r, c + r)), \\
  T_7 \coloneqq f((a - r, b + r, c + r)), \quad
  T_8 \coloneqq f((a + r, b + r, c + r))
\end{gather*}
の8つである.ここで, \(f\) は式\eqref{eq:get-coordinates}
の \(f\) である.
}

% physics/collision-flow/3.tex
{%
% 抽出された8つのタイルの位置とその間のタイルの位置についてタイルが存在するものを取り出す
\subsubsection{\sec@collision@flow{3}}
\label{sec:collision-flow-3}

前節 \ref{sec:collision-flow-2} のタイル \(T_\lambda \, (\lambda = 1, \dots, 8)\)
について,そのタイルらに囲まれた(\(T_\lambda\) を含む)タイルの集合は,
\(S \coloneqq \{[i, j, k] \mid
  i_- \le i \le i_+, \, 
  j_- \le j \le j_+, \, 
  k_- \le k \le k_+ \}\)
と表せる.
このうち,タイルが存在するものは,
\(\{ [i, j, k] | [i, j, k] \in S, \, \verb|tile_state|[i][j][k] = 1\}\)
である.
ここで,
\verb|tile_state| は
ファイル \verb|src/resources.rs| (変更の可能性あり)
の \verb|TileState| のインスタンスである.

}



\subsubsection{\sec@collision@flow{4}}
\label{sec:collision-flow-4}

\subsubsection{\sec@collision@flow{5}}
\label{sec:collision-flow-5}

\makeatother
}
