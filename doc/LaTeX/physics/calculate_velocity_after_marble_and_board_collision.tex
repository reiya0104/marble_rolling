
\subsection{%
ビー玉と天板の衝突後のビー玉の速度を算出する
\label{calculate_velocity_after_marble_and_board_collision}
}

% TODO: 後ほど TikZ で図を描画する

{%
\makeatletter
\newcommand{\velocity}{\bm{v}}
\newcommand{\v@marble}{\velocity_{\rm marble}}
\newcommand{\v@marble@prime}{\v@marble^\prime}
\newcommand{\v@board}{\velocity_{\rm board}}
\newcommand{\v@marble@from@board}{\velocity_{{\rm marble} \leftarrow {\rm board}}}
\newcommand{\v@marble@from@board@prime}{\v@marble@from@board^\prime}

\newcommand{\v@m}{\velocity_{\rm m}}
\newcommand{\v@m@prime}{\v@m^\prime}
\newcommand{\v@b}{\velocity_{\rm b}}

\newcommand{\v@m@from@b}{\velocity_{{\rm m} \leftarrow {\rm b}}}
\newcommand{\v@m@from@b@n}{{\v@m@from@b}^{\n}}
\newcommand{\v@m@from@b@n@prime}{{\v@m@from@b}^{\n^\prime}}

\newcommand{\v@m@prime@from@b}{\velocity^\prime_{{\rm m} \leftarrow {\rm b}}}
\newcommand{\v@m@prime@from@b@n}{{\v@m@prime@from@b}^{\n}}
\newcommand{\v@m@prime@from@b@n@prime}{{\v@m@prime@from@b}^{\n^\prime}}


\newcommand{\n}{\bm{n}}

\begin{itembox}[l]{\bf 問題}
  速度 \(\v@marble\) で動くビー玉(marble)と,
  速度 \(\v@board\) で動く天板(board)が衝突したとする.
  衝突後のビー玉の速度 \(\v@marble@prime\) を求めよ.

  ここで,天板の単位法線ベクトル(衝突面を上向きにしたもの)は \(\n\),
  天板とビー玉間のはね返り定数を \(e\) とする. 

  ただし,天板の表面は滑らかであるとし,衝突時に影響を受けないものとする.
\end{itembox}

\noindent
{\bf 結論}
\begin{align}
  \label{eq:v@marble@prime}
  \v@marble@prime = \v@marble - (1 + e) ((\v@marble - \v@board) \cdot \n) \, \n
\end{align}
\vskip\baselineskip
\noindent
{\bf 方針}

方針としては,
\begin{enumerate}[(1)]
  \item 静止している系から考えるのは難しいので,天板から見た系を考える.
  \item 天板から見た系におけるビー玉の速度を求める.
  \item 天板から見た系において衝突後のビー玉の速度を求める.
  \item 静止している系における衝突後のビー玉の速度を求める.
\end{enumerate}
として求める.
\vskip\baselineskip
\noindent
{\bf 解答}

便宜上
\(\v@m \coloneqq \v@marble, \,
  \v@b \coloneqq \v@board \)
とおく.

\begin{enumerate}[(1)]
  \item 天板から見た系を考える.
  \item 天板から見た系におけるビー玉の速度を \(\v@m@from@b\) とする.
  このとき,\(\v@m@from@b\) は
  天板に対するビー玉の相対速度であるから,
  \begin{align}
    \label{eq:v@m@from@b}
    \v@m@from@b = \v@m - \v@b
  \end{align}
  である.
  \item 天板から見た系において衝突後のビー玉の速度を \(\v@m@prime@from@b\) とする.
  
  \(\v@m@from@b\) を天板の鉛直方向成分
  \(\v@m@from@b@n\)
  と平行方向成分
  \(\v@m@from@b@n@prime\)
  に分解すると,
  \begin{align*}
    \v@m@from@b@n &= (\v@m@from@b \cdot \n) \, \n \\
    \v@m@from@b@n@prime &= \v@m@from@b - \v@m@from@b@n \\
    &= \v@m@from@b - (\v@m@from@b \cdot \n) \, \n
  \end{align*}
  である.

  天板との衝突後,ビー玉の速度の鉛直方向の成分を
  \(\v@m@prime@from@b@n\) とすると
  \(\v@m@prime@from@b@n = -e \, \v@m@from@b@n\) である.
  平行方向成分は変わらず
  \(\v@m@from@b@n@prime\)
  であるから,
  \begin{align}
    \v@m@prime@from@b &= \v@m@prime@from@b@n + \v@m@from@b@n@prime \notag \\
    &= -e \, \v@m@from@b@n + \v@m@from@b@n@prime \notag \\
    &= -e ((\v@m@from@b \cdot \n) \, \n) + (\v@m@from@b - (\v@m@from@b \cdot \n) \, \n) \notag \\
    &=
    \label{eq:v@m@prime@from@b}
    \v@m@from@b - (1 + e) (\v@m@from@b \cdot \n) \, \n
  \end{align}
  である.
  \item 静止している系における衝突後のビー玉の速度 \(\v@marble@prime\) は,
  \(\v@m@prime@from@b\) と \(\v@b\) で表すと,
  \(\v@marble@prime = \v@m@prime@from@b + \v@b \)
  である.
  これと式 \eqref{eq:v@m@from@b}, \eqref{eq:v@m@prime@from@b} を用いると,
  \begin{align*}
    \v@marble@prime &= (\v@m@from@b - (1 + e) (\v@m@from@b \cdot \n) \, \n) + \v@b \\
    &= ((\v@m - \v@b) - (1 + e) ((\v@m - \v@b) \cdot \n) \, \n) + \v@b \\
    &= \v@m - (1 + e) ((\v@m - \v@b) \cdot \n) \, \n
  \end{align*}
  である.
  
  したがって,
  \begin{align}
    \v@marble@prime = \v@marble - (1 + e) ((\v@marble - \v@board) \cdot \n) \, \n
    \tag{\ref{eq:v@marble@prime}}
  \end{align}
  が成り立つ.
\end{enumerate}

\makeatother
}
