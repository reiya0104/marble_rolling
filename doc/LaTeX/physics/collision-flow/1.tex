% physics/collision-flow/1.tex

{%
\makeatletter

\subsubsection{\sec@collision@flow{1}}
\label{sec:collision-flow-1}

タイルの(直交)座標系の基本ベクトルを \(\e@tile@x, \e@tile@y, \e@tile@z\)
とおく.
また,タイルの座標系の原点 \(\O@tile\) のワールドの座標系における位置ベクトルを
\(\v@O@tile\) とする.

ここで,ビー玉について,ビー玉のある点をPとし,
ワールドの座標系における位置ベクトルを \(\v@P\),
タイルの座標系における位置ベクトルを \(\v@tile@P\)
とする.

このとき,\(\v@tile@P\) は \(\e@tile@x, \e@tile@y, \e@tile@z, \v@O@tile, \v@P\) を用いて
\begin{align}
  \v@tile@P =
  \begin{pmatrix}
    \e@tile@x & \e@tile@y & \e@tile@z
  \end{pmatrix}^{-1}
  (\v@P - \v@O@tile)
\end{align}
と表せる.

% physics/collision-flow/1_tikzpicture.tex
{%

% xyz軸の長さ
\newcommand{\@size@x}{7}
\newcommand{\@size@y}{5}
\newcommand{\@size@z}{3}

% 点P の座標
\newcommand{\@x}{3}
\newcommand{\@y}{4}
\newcommand{\@z}{2}

% タイルの座標の原点 \O@tile の座標
\newcommand{\O@tile@x}{6.5}
\newcommand{\O@tile@y}{1}
\newcommand{\O@tile@z}{2.5}

% \e@tile@xの座標
\newcommand{\e@tile@x@x}{1/3}
\newcommand{\e@tile@x@y}{1/2}
\newcommand{\e@tile@x@z}{0}
% \e@tile@yの座標
\newcommand{\e@tile@y@x}{-1/4}
\newcommand{\e@tile@y@y}{1/2}
\newcommand{\e@tile@y@z}{-1/3}
% \e@tile@zの座標
\newcommand{\e@tile@z@x}{-3/4}
\newcommand{\e@tile@z@y}{0}
\newcommand{\e@tile@z@z}{1/3}

\begin{center}
  \begin{tikzpicture}
    % 点
    % ワールド座標系
    \coordinate (O) at (0,0); % 原点 O
    \coordinate (e_x) at (1,0);
    \coordinate (e_y) at (0,1);
    \coordinate (e_z) at ($5/6*(-1/2,{-sqrt(3)/2})$);
    \coordinate (P) at ($\@x*(e_x)+\@y*(e_y)+\@z*(e_z)$);

    % タイルの座標系
    \coordinate (O_tile) at ($\O@tile@x*(e_x)+\O@tile@y*(e_y)+\O@tile@z*(e_z)$);
    \coordinate (e_tile_x) at ($\e@tile@x@x*(e_x)+\e@tile@x@y*(e_y)+\e@tile@x@z*(e_z)$);
    \coordinate (e_tile_y) at ($\e@tile@y@x*(e_x)+\e@tile@y@y*(e_y)+\e@tile@y@z*(e_z)$);
    \coordinate (e_tile_z) at ($\e@tile@z@x*(e_x)+\e@tile@z@y*(e_y)+\e@tile@z@z*(e_z)$);
    
    % \v@tile@P の計算
    \tikzmath{%
      % 3*3行列の逆行列を求める
      % 行列に代入
      %
      % [\m@a@a \m@a@b \m@a@c]
      % [\m@b@a \m@b@b \m@b@c]
      % [\m@c@a \m@c@b \m@c@c]
      %
      \m@a@a = \e@tile@x@x;
      \m@a@b = \e@tile@y@x;
      \m@a@c = \e@tile@z@x;
      \m@b@a = \e@tile@x@y;
      \m@b@b = \e@tile@y@y;
      \m@b@c = \e@tile@z@y;
      \m@c@a = \e@tile@x@z;
      \m@c@b = \e@tile@y@z;
      \m@c@c = \e@tile@z@z;
      %
      % 行列式
      \@det = \m@a@a * \m@b@b * \m@c@c
        + \m@a@b * \m@b@c * \m@c@a
        + \m@a@c * \m@b@a * \m@c@b
        - \m@a@a * \m@b@c * \m@c@b
        - \m@a@b * \m@b@a * \m@c@c
        - \m@a@c * \m@b@b * \m@c@a;
      %
      % 逆行列
      \i@a@a =  (\m@b@b * \m@c@c - \m@b@c * \m@c@b) / \@det;
      \i@a@b = -(\m@a@b * \m@c@c - \m@a@c * \m@c@b) / \@det;
      \i@a@c =  (\m@a@b * \m@b@c - \m@a@c * \m@b@b) / \@det;
      \i@b@a = -(\m@b@a * \m@c@c - \m@b@c * \m@c@a) / \@det;
      \i@b@b =  (\m@a@a * \m@c@c - \m@a@c * \m@c@a) / \@det;
      \i@b@c = -(\m@a@a * \m@b@c - \m@a@c * \m@b@a) / \@det;
      \i@c@a =  (\m@b@a * \m@c@b - \m@b@b * \m@c@a) / \@det;
      \i@c@b = -(\m@a@a * \m@c@b - \m@a@b * \m@c@a) / \@det;
      \i@c@c =  (\m@a@a * \m@b@b - \m@a@b * \m@b@a) / \@det;
      %
      % ベクトルに代入
      \v@x = \@x - \O@tile@x;
      \v@y = \@y - \O@tile@y;
      \v@z = \@z - \O@tile@z;
      %
      % 逆行列 * ベクトル
      \iv@x = \i@a@a * \v@x + \i@a@b * \v@y + \i@a@c * \v@z;
      \iv@y = \i@b@a * \v@x + \i@b@b * \v@y + \i@b@c * \v@z;
      \iv@z = \i@c@a * \v@x + \i@c@b * \v@y + \i@c@c * \v@z;
      %
      % 数値を代入
      \v@tile@P@x = \iv@x;
      \v@tile@P@y = \iv@y;
      \v@tile@P@z = \iv@z;
    }
    \coordinate (v_tile_P) at ($\v@tile@P@x*(e_tile_x)+\v@tile@P@y*(e_tile_y)+\v@tile@P@z*(e_tile_z)$);

    % 点の描画
    \fill (O) circle (0.06) node[above left]{O}; % 点O
    % xyz軸の描画
    \draw[->,>=stealth,semithick] (O)--($\@size@x*(e_x)$)node[above]{$x$}; % x軸
    \draw[->,>=stealth,semithick] (O)--($\@size@y*(e_y)$)node[left]{$y$}; % y軸
    \draw[->,>=stealth,semithick] (O)--($\@size@z*(e_z)$)node[above left]{$z$}; % z軸

    % 基本ベクトルの描画
    % \draw[->,>=stealth,very thick] (O)--(e_x)node[above]{$\e@x$}; % e_x
    % \draw[->,>=stealth,very thick] (O)--(e_y)node[left]{$\e@y$}; % e_y
    % \draw[->,>=stealth,very thick] (O)--(e_z)node[above left]{$\e@z$}; % e_z

    % \v@P の描画
    \draw[->,>=stealth,very thick] (O)--(P)node[xshift=-20,yshift=-40]{$\v@P$}; % v_P

    \fill (P) circle (0.06) node[above left]{P}; % 点P
    % 点Pのx座標からの点線
    \draw[dashed, thin] ($\@x*(e_x)$)--($\@x*(e_x)+\@y*(e_y)$);
    \draw[dashed, thin] ($\@x*(e_x)$)--($\@x*(e_x)+\@z*(e_z)$);
    % 点Pのy座標からの点線
    \draw[dashed, thin] ($\@y*(e_y)$)--($\@y*(e_y)+\@x*(e_x)$);
    \draw[dashed, thin] ($\@y*(e_y)$)--($\@y*(e_y)+\@z*(e_z)$);
    % 点Pのz座標からの点線
    \draw[dashed, thin] ($\@z*(e_z)$)--($\@z*(e_z)+\@y*(e_y)$);
    \draw[dashed, thin] ($\@z*(e_z)$)--($\@z*(e_z)+\@x*(e_x)$);
    % 点Pへの点線
    \draw[dashed, thin] ($\@x*(e_x)+\@y*(e_y)$)--(P);
    \draw[dashed, thin] ($\@y*(e_y)+\@z*(e_z)$)--(P);
    \draw[dashed, thin] ($\@z*(e_z)+\@x*(e_x)$)--(P);


    \fill (O_tile) circle (0.06) node[right]{$\O@tile$}; % 点 \O@tile の描画
    \draw[->,>=stealth,very thick] (O)--(O_tile)node[left,xshift=-40,yshift=15]{$\v@O@tile$}; % v_P

    \draw[dashed, thin] (O)--($\O@tile@x*(e_x)$); % 点 \O@tile の描画
    % 点 \O@tile のx座標からの点線
    \draw[dashed, thin] ($\O@tile@x*(e_x)$)--($\O@tile@x*(e_x)+\O@tile@y*(e_y)$);
    \draw[dashed, thin] ($\O@tile@x*(e_x)$)--($\O@tile@x*(e_x)+\O@tile@z*(e_z)$);
    % 点 \O@tile のy座標からの点線
    \draw[dashed, thin] ($\O@tile@y*(e_y)$)--($\O@tile@y*(e_y)+\O@tile@x*(e_x)$);
    \draw[dashed, thin] ($\O@tile@y*(e_y)$)--($\O@tile@y*(e_y)+\O@tile@z*(e_z)$);
    % 点 \O@tile のz座標からの点線
    \draw[dashed, thin] ($\O@tile@z*(e_z)$)--($\O@tile@z*(e_z)+\O@tile@y*(e_y)$);
    \draw[dashed, thin] ($\O@tile@z*(e_z)$)--($\O@tile@z*(e_z)+\O@tile@x*(e_x)$);
    % 点Pへの点線
    \draw[dashed, thin] ($\O@tile@x*(e_x)+\O@tile@y*(e_y)$)--(O_tile);
    \draw[dashed, thin] ($\O@tile@y*(e_y)+\O@tile@z*(e_z)$)--(O_tile);
    \draw[dashed, thin] ($\O@tile@z*(e_z)+\O@tile@x*(e_x)$)--(O_tile);
    
    % タイルの座標系の基本ベクトル
    \draw[->,>=stealth,very thick, blue] (O_tile)--($(O_tile)+(e_tile_x)$)node[right]{$\e@tile@x$}; % e_tile_x
    \draw[->,>=stealth,very thick, blue] (O_tile)--($(O_tile)+(e_tile_y)$)node[above left]{$\e@tile@y$}; % e_tile_y
    \draw[->,>=stealth,very thick, blue] (O_tile)--($(O_tile)+(e_tile_z)$)node[below]{$\e@tile@z$}; % e_tile_z

    % v_tile_P の描画
    % \draw[->,>=stealth,very thick] (O_tile)--(P)node[left,xshift=-20,yshift=-7]{$\v@tile@P$}; % v_tile_P
    \draw[->,>=stealth,very thick, blue](O_tile)--($(O_tile)+(v_tile_P)$)
      node[right,xshift=23,yshift=-20]{$\v@tile@P$}; % v_tile_P
    
    % O_tile から座標への点線
    \draw[dashed, thin, blue] (O_tile)--($(O_tile)+\v@tile@P@x*(e_tile_x)$);
    \draw[dashed, thin, blue] (O_tile)--($(O_tile)+\v@tile@P@y*(e_tile_y)$);
    \draw[dashed, thin, blue] (O_tile)--($(O_tile)+\v@tile@P@z*(e_tile_z)$);
    % タイルの座標系での点Pのx座標からの点線
    \draw[dashed, thin, blue] ($(O_tile)+\v@tile@P@x*(e_tile_x)$)--($(O_tile)+\v@tile@P@x*(e_tile_x)+\v@tile@P@y*(e_tile_y)$);
    \draw[dashed, thin, blue] ($(O_tile)+\v@tile@P@x*(e_tile_x)$)--($(O_tile)+\v@tile@P@x*(e_tile_x)+\v@tile@P@z*(e_tile_z)$);
    % タイルの座標系での点Pのy座標からの点線
    \draw[dashed, thin, blue] ($(O_tile)+\v@tile@P@y*(e_tile_y)$)--($(O_tile)+\v@tile@P@y*(e_tile_y)+\v@tile@P@x*(e_tile_x)$);
    \draw[dashed, thin, blue] ($(O_tile)+\v@tile@P@y*(e_tile_y)$)--($(O_tile)+\v@tile@P@y*(e_tile_y)+\v@tile@P@z*(e_tile_z)$);
    % タイルの座標系での点Pのz座標からの点線
    \draw[dashed, thin, blue] ($(O_tile)+\v@tile@P@z*(e_tile_z)$)--($(O_tile)+\v@tile@P@z*(e_tile_z)+\v@tile@P@y*(e_tile_y)$);
    \draw[dashed, thin, blue] ($(O_tile)+\v@tile@P@z*(e_tile_z)$)--($(O_tile)+\v@tile@P@z*(e_tile_z)+\v@tile@P@x*(e_tile_x)$);
    % 点Pへの点線
    \draw[dashed, thin, blue] ($(O_tile)+\v@tile@P@x*(e_tile_x)+\v@tile@P@y*(e_tile_y)$)--(P);
    \draw[dashed, thin, blue] ($(O_tile)+\v@tile@P@y*(e_tile_y)+\v@tile@P@z*(e_tile_z)$)--(P);
    \draw[dashed, thin, blue] ($(O_tile)+\v@tile@P@z*(e_tile_z)+\v@tile@P@x*(e_tile_x)$)--(P);
  \end{tikzpicture}
\end{center}
}


}
