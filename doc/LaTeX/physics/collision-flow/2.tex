% physics/collision-flow/2.tex
{%
% ビー玉の各軸の最大最小の座標に基づくタイルの位置を抽出
\subsubsection{\sec@collision@flow{2}}
\label{sec:collision-flow-2}

以下はタイルの座標系として考える.

本題に入る前に,タイルの構成と,点とタイルの位置関係について述べる.

\noindent
{\bf タイルの構成}

タイルは直方体でできており,下の図のように敷き詰められている.
ここで, \(s = \verb|TILE_SHORT_WIDTH|, \,
l = \verb|TILE_LONG_WIDTH|\) とする.

% physics/collision-flow/2_tikzpicture.tex
{%

% xyz軸の長さ
\newcommand{\@size@x}{7.5}
\newcommand{\@size@y}{3}
\newcommand{\@size@z}{3.5}

\newcommand{\s@length}{0.8}
\newcommand{\l@length}{2.0}

\newcommand{\max@i}{4}
\newcommand{\max@j}{1}
\newcommand{\max@k}{1}

\begin{center}
  \begin{tikzpicture}
    % 点
    % ワールド座標系
    \coordinate (O_tile) at (0,0);
    \coordinate (e_tile_x) at (1,0);
    \coordinate (e_tile_y) at (0,1);
    \coordinate (e_tile_z) at ($5/6*(-1/2,{-sqrt(3)/2})$);
    \coordinate (O_tile_draw) at ($(O_tile)+(0, -\s@length)$);

    % 座標系の描画
    \fill (O_tile) circle (0.06) node[below right]{$\O@tile$};
    \draw[->,>=stealth,very thick] (O_tile)--($\@size@x*(e_tile_x)$)node[above]{$\x@tile$}; % x座標
    \draw[->,>=stealth,very thick] (O_tile)--($\@size@y*(e_tile_y)$)node[left]{$\y@tile$}; % y座標
    \draw[->,>=stealth,very thick] (O_tile)--($\@size@z*(e_tile_z)$)node[below]{$\z@tile$}; % z座標
    
    \tikzmath{%
      \max@x@cood = div(\max@i+2,2)*\s@length+div(\max@i+1,2)*\l@length;
      \max@y@cood = div(\max@j+2,2)*\s@length+div(\max@j+1,2)*\l@length;
      \max@z@cood = div(\max@k+2,2)*\s@length+div(\max@k+1,2)*\l@length;
    }
    \draw[dotted] ($(O_tile)+(0, -\s@length)$)--($(O_tile)+(0, -\s@length)+\@size@x*(e_tile_x)$);
    \draw[dotted] ($(O_tile)+(0, -\s@length)$)--($(O_tile)+(0, -\s@length)+\@size@y*(e_tile_y)$);
    \draw[dotted] ($(O_tile)+(0, -\s@length)$)--($(O_tile)+(0, -\s@length)+\@size@z*(e_tile_z)$);
    % 繰り返し
    \foreach \i in {0, ..., \max@i}
    {%
      \foreach \j in {0, ..., \max@j}
      {%
        \foreach \k in {0, ..., \max@k}
        {%
          % タイルの枠組みの描画 
          \tikzmath{%
            \tile@x = div(\i+1,2)*\s@length+div(\i,2)*\l@length;
            \tile@y = div(\j+1,2)*\s@length+div(\j,2)*\l@length;
            \tile@z = div(\k+1,2)*\s@length+div(\k,2)*\l@length;
            %
            \tile@next@x = div(\i+2,2)*\s@length+div(\i+1,2)*\l@length;
            \tile@next@y = div(\j+2,2)*\s@length+div(\j+1,2)*\l@length;
            \tile@next@z = div(\k+2,2)*\s@length+div(\k+1,2)*\l@length;
            %
            % 表示する用の値
            \tile@node@x@s@count = div(\i+2,2);
            \tile@node@x@l@count = div(\i+1,2);
            \tile@node@y@s@count = div(\j,2);
            \tile@node@y@l@count = div(\j+1,2);
            \tile@node@z@s@count = div(\k+2,2);
            \tile@node@z@l@count = div(\k+1,2);
          }

          % タイルの影の描画
          \ifthenelse{\(\j=0 \AND \k=1\) \OR \(\j=0 \AND \i=3\)}{
            \shade ($(O_tile_draw)+\tile@next@x*(e_tile_x)+\tile@next@y*(e_tile_y)+\tile@z*(e_tile_z)$)
              --($(O_tile_draw)+\tile@next@x*(e_tile_x)+\tile@next@y*(e_tile_y)+\tile@next@z*(e_tile_z)$)
              --($(O_tile_draw)+\tile@next@x*(e_tile_x)+\tile@y*(e_tile_y)+\tile@next@z*(e_tile_z)$)
              --($(O_tile_draw)+\tile@next@x*(e_tile_x)+\tile@y*(e_tile_y)+\tile@z*(e_tile_z)$)
              --cycle;
            \shade ($(O_tile_draw)+\tile@next@x*(e_tile_x)+\tile@next@y*(e_tile_y)+\tile@z*(e_tile_z)$)
              --($(O_tile_draw)+\tile@next@x*(e_tile_x)+\tile@next@y*(e_tile_y)+\tile@next@z*(e_tile_z)$)
              --($(O_tile_draw)+\tile@x*(e_tile_x)+\tile@next@y*(e_tile_y)+\tile@next@z*(e_tile_z)$)
              --($(O_tile_draw)+\tile@x*(e_tile_x)+\tile@next@y*(e_tile_y)+\tile@z*(e_tile_z)$)
              --cycle;
            \shade ($(O_tile_draw)+\tile@x*(e_tile_x)+\tile@next@y*(e_tile_y)+\tile@next@z*(e_tile_z)$)
              --($(O_tile_draw)+\tile@next@x*(e_tile_x)+\tile@next@y*(e_tile_y)+\tile@next@z*(e_tile_z)$)
              --($(O_tile_draw)+\tile@next@x*(e_tile_x)+\tile@y*(e_tile_y)+\tile@next@z*(e_tile_z)$)
              --($(O_tile_draw)+\tile@x*(e_tile_x)+\tile@y*(e_tile_y)+\tile@next@z*(e_tile_z)$)
              --cycle;
          }{};

          \ifthenelse{\j=0 \AND \k=0}{
            \draw[dotted] ($(O_tile_draw)+\tile@next@x*(e_tile_x)+\@size@y*(e_tile_y)$)
              --($(O_tile_draw)+\tile@next@x*(e_tile_x)$)
              --($(O_tile_draw)+\tile@next@x*(e_tile_x)+\@size@z*(e_tile_z)$);
              \fill($(O_tile)+\tile@next@x*(e_tile_x)$) circle (0.06) node[%
              \ifodd\i below\else above\fi
              ]{$\tile@node@x@s@count \, s + \tile@node@x@l@count \, l$};
          }{};
          \ifthenelse{\k=0 \AND \i=0}{
            \draw[dotted] ($(O_tile_draw)+\tile@next@y*(e_tile_y)+\@size@z*(e_tile_z)$)
              --($(O_tile_draw)+\tile@next@y*(e_tile_y)$)
              --($(O_tile_draw)+\tile@next@y*(e_tile_y)+\@size@x*(e_tile_x)$);
            \fill($(O_tile_draw)+\tile@next@y*(e_tile_y)$) circle (0.06) node[left]{$\tile@node@y@s@count \, s + \tile@node@y@l@count \, l$};
          }{};
          \ifthenelse{\i=0 \AND \j=0}{
            \draw[dotted] ($(O_tile_draw)+\tile@next@z*(e_tile_z)+\@size@x*(e_tile_x)$)
              --($(O_tile_draw)+\tile@next@z*(e_tile_z)$)
              --($(O_tile_draw)+\tile@next@z*(e_tile_z)+\@size@y*(e_tile_y)$);
            \fill($(O_tile)+\tile@next@z*(e_tile_z)$) circle (0.06) node[left]{$\tile@node@z@s@count \, s + \tile@node@z@l@count \, l$};
          }{};
          \draw[dotted] ($(O_tile_draw)+\tile@next@x*(e_tile_x)+\tile@next@y*(e_tile_y)+\tile@z*(e_tile_z)$)
            --($(O_tile_draw)+\tile@next@x*(e_tile_x)+\tile@next@y*(e_tile_y)+\tile@next@z*(e_tile_z)$)
            --($(O_tile_draw)+\tile@next@x*(e_tile_x)+\tile@y*(e_tile_y)+\tile@next@z*(e_tile_z)$);
          
          \draw[dotted] ($(O_tile_draw)+\tile@x*(e_tile_x)+\tile@next@y*(e_tile_y)+\tile@next@z*(e_tile_z)$)
            --($(O_tile_draw)+\tile@next@x*(e_tile_x)+\tile@next@y*(e_tile_y)+\tile@next@z*(e_tile_z)$);
          
        }
      }
    }
  \end{tikzpicture}
\end{center}
}


この図のように,
\(\x@tile, \y@tile, \z@tile\) 軸のそれぞれの方向における \(i, j, k\) 番目(\(i, j, k = 0, 1, 2, \dots\))のタイルは,

\begin{align}
  \label{eq:tile-aria}
  \begin{alignedat}{9}
    &\x@tile \, \text{軸:} &\quad & 
    &\biggl\lfloor \frac{i + 1}{2} \biggr\rfloor \, & s + & \biggl\lfloor \frac{i}{2} \biggr\rfloor \, & l
    && \le \x@tile
    < & \biggl\lfloor \frac{i + 2}{2} \biggr\rfloor \, & s + & \biggl\lfloor \frac{i + 1}{2} \biggr\rfloor \, & l \\
    &\y@tile \, \text{軸:} & & 
    &\biggl\lfloor \frac{j - 1}{2} \biggr\rfloor \, & s + & \biggl\lfloor \frac{j}{2} \biggr\rfloor \, & l
    && \le \y@tile
    < & \biggl\lfloor \frac{j}{2} \biggr\rfloor \, & s + & \biggl\lfloor \frac{j + 1}{2} \biggr\rfloor \, & l \\
    &\z@tile \, \text{軸:} & & 
    &\biggl\lfloor \frac{k + 1}{2} \biggr\rfloor \, & s + & \biggl\lfloor \frac{k}{2} \biggr\rfloor \, & l
    && \le \z@tile
    < & \biggl\lfloor \frac{k + 2}{2} \biggr\rfloor \, & s + & \biggl\lfloor \frac{k + 1}{2} \biggr\rfloor \, & l
  \end{alignedat}
\end{align}

の領域に配置される.
ここで,
\(\lfloor x \rfloor\) は \(n \le x < n + 1\) となる整数 \(n\) のことである.
今後,この領域にあるタイルのことを \([i, j, k]\) と表すとする.

例えば, \([1, 2, 3]\) は,
\(( s \le \x@tile < s + l \ \text{かつ} \ s \le \y@tile < s + l \ \text{かつ} \ 2 \, s + l \le \z@tile < 2 \, s + 2 \, l )\)
という領域のタイルを表す.

\noindent
{\bf 点とタイルの位置関係}

このようなタイルの構成について,ある点P \((a, b, c)\) がどのタイル \([i, j, k]\) に含まれているのかを考えたい.
これは,上の 式\eqref{eq:tile-aria} から逆算すると,
\(i, j, k\) はそれぞれ

\begin{align}
  \begin{aligned}
    i &= 2 \biggl\lfloor \frac{a}{s + l} \biggr\rfloor
      + \biggl\lfloor \frac{a - s}{s + l} - \biggl\lfloor \frac{a}{s + l} \biggr\rfloor \biggr\rfloor
      + 1 \eqqcolon f_i(a) \\
    j &= 2 \biggl\lfloor \frac{b + s}{s + l} \biggr\rfloor
      + \biggl\lfloor \frac{b}{s + l} - \biggl\lfloor \frac{b + s}{s + l} \biggr\rfloor \biggr\rfloor
      + 1 \eqqcolon f_j(b) \\
    k &= 2 \biggl\lfloor \frac{c}{s + l} \biggr\rfloor
      + \biggl\lfloor \frac{c - s}{s + l} - \biggl\lfloor \frac{c}{s + l} \biggr\rfloor \biggr\rfloor
      + 1 \eqqcolon f_k(c)
  \end{aligned}
\end{align}
となる.
このように,点P \((a, b, c)\) からその点を含むタイル \([i, j, k]\)
を返す関数を
\begin{align}
  \label{eq:get-coordinates}
  f((a, b, c)) = [f_i(a), f_j(b), f_k(c)]
\end{align}
とする.

求め方:
\begin{quotation}
  \(i\) を求める.\(i\) は0以上の整数であるから,
  ある非負整数 \(n\) を用いて
  \(i = 2 n \, \text{または} \, i = 2 (n + 1)\) と表せる.
  このとき,式\eqref{eq:tile-aria} の \(\x@tile\) の範囲はそれぞれ
  \begin{alignat*}{3}
    &i = 2 n \text{のとき}: & n (s + l) &\le \x@tile < n (s + l) + s \\
    &i = 2 n + 1 \text{のとき}: \quad & n (s + l) + s &\le \x@tile < (n + 1) (s + l)
  \end{alignat*}
  となる.各辺を \(s + l\) で割ると,\(s + l > 0\) から,
  \begin{alignat*}{3}
    &i = 2 n \text{のとき}: & n &\le \frac{\x@tile}{s + l} < n + \frac{s}{s + l} \\
    &i = 2 n + 1 \text{のとき}: \quad & n + \frac{s}{s + l} &\le \frac{\x@tile}{s + l} < n + 1
  \end{alignat*}
  がいえる.これより,\(a\) について
  \(n^\prime \le \frac{a}{s + l} < n^\prime + \frac{s}{s + l}\)
  を満たすような非負整数 \(n^\prime\) が存在すれば,
  この \(n^\prime\) について \(i = 2 n^\prime\) が成り立ち,
  \(n^\prime + \frac{s}{s + l} \le \frac{a}{s + l} < n^\prime + 1\)
  を満たすような非負整数 \(n^\prime\) が存在すれば,
  この \(n^\prime\) について \(i = 2 n^\prime + 1\) が成り立つ.
  任意の \(a\) と \(n^\prime\) について \(\frac{a}{s + l} < n^\prime + \frac{s}{s + l} \le \frac{a}{s + l}\)
  は成立しないので,
  上の二つはどちらかしか成り立たない.

  \(m \coloneqq \bigl\lfloor \frac{a}{s + l} \bigr\rfloor\)
  とすると,\( m \le \frac{a}{s + l} < m + 1\) より,
  \(n^\prime = m\) である.
  このとき,
  \begin{gather*}
    i = 2 m + g(a)\\
    \left(
    \text{ここで,} \, g(a) = \left\{\,\begin{alignedat}{3}
      &0 \quad \Bigl( (m \le ) &&\frac{a}{s + l} < m + \frac{s}{s + l} \, \text{のとき}\Bigr)\\
      &1 \quad \Bigl( (m + 1 > ) &&\frac{a}{s + l} \ge m + \frac{s}{s + l} \, \text{のとき}\Bigr)
    \end{alignedat}
    \right.
    \right)
  \end{gather*}
  が成り立つ.
  \(g(a)\) は複数の表示があるが,ここではガウス記号(床関数)を用いると\\
  \(g(a) = \bigl\lfloor \frac{a}{s + l} - \bigl(m + \frac{s}{s + l} \bigr)\bigr\rfloor + 1
  = \bigl\lfloor \frac{a - s}{s + l} - m \bigr\rfloor + 1\)
  と表せる.
  よって,
  \(i = 2 \bigl\lfloor \frac{a}{s + l} \bigr\rfloor
  + \bigl\lfloor \frac{a - s}{s + l} - \bigl\lfloor \frac{a}{s + l} \bigr\rfloor \bigr\rfloor
  + 1\)
  となる.\(j, k\) についても同様にして求められる.
\end{quotation}

\vskip\baselineskip

ここから本題に入る.
ビー玉の半径を \(r\) とし,
ビー玉の中心がタイルの座標系において \((a, b, c)\)
の位置にあるとする.

このとき,各軸の最大最小の座標に基づくタイルの位置を求める.
\(\x@tile\) 軸における最大・最小値はそれぞれ
\(a + r, a - r\) である.
同様に,\(\y@tile\) 軸では \(b + r, b - r\),
\(\z@tile\) 軸では \(c + r, c - r\) である.
これより,求めるタイルは
\begin{gather*}
  T_1 \coloneqq f((a - r, b - r, c - r)), \quad
  T_2 \coloneqq f((a + r, b - r, c - r)), \\
  T_3 \coloneqq f((a - r, b + r, c - r)), \quad
  T_4 \coloneqq f((a + r, b + r, c - r)), \\
  T_5 \coloneqq f((a - r, b - r, c + r)), \quad
  T_6 \coloneqq f((a + r, b - r, c + r)), \\
  T_7 \coloneqq f((a - r, b + r, c + r)), \quad
  T_8 \coloneqq f((a + r, b + r, c + r))
\end{gather*}
の8つである.ここで, \(f\) は式\eqref{eq:get-coordinates}
の \(f\) である.
}
