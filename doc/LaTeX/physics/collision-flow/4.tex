% physics/collision-flow/4.tex
{%
% 存在するタイルについてそのタイルとビー玉との距離を計算
\subsubsection{\sec@collision@flow{4}}
\label{sec:collision-flow-4}

\begin{dfn}
  あるタイルの境界の集合を \(S_{\text{タイル}}\), 
  ビー玉の境界の集合を \(S_{\text{ビー玉}}\) とおく.
  このとき,タイルとビー玉の {\bf 距離} \(d\)
  を,
  \[d = \min\{ |p - q| \mid p \in S_{\text{タイル}}, q \in S_{\text{ビー玉}} \}\]
  と定義する(すなわち、タイルとビー玉の最短距離のことである).
\end{dfn}


ビー玉の位置を
\((a, b, c)\),
衝突判定を行うタイルを \([i, j, k]\)
とする.
ビー玉の半径を \(r\) とする.

式 \eqref{eq:tile-aria}
について,
\(\x@tile, \y@tile, \z@tile\) 軸
におけるタイルが存在する範囲の下限と上限をそれぞれ
\begin{align*}
  &\x@tile_- \coloneqq \inf \bigl( X^\prime_{[i, j, k]} \bigr)
  = \biggl\lfloor \frac{i + 1}{2} \biggr\rfloor \, s + \biggl\lfloor \frac{i}{2} \biggr\rfloor  \, l, \quad
  \x@tile_+\coloneqq \sup \bigl( X^\prime_{[i, j, k]} \bigr)
  = \biggl\lfloor \frac{i + 2}{2} \biggr\rfloor \, s + \biggl\lfloor \frac{i + 1}{2} \biggr\rfloor \, l \\
  &\y@tile_- \coloneqq \inf \bigl( Y^\prime_{[i, j, k]} \bigr)
  = \biggl\lfloor \frac{j - 1}{2} \biggr\rfloor \, s + \biggl\lfloor \frac{j}{2} \biggr\rfloor  \, l, \quad
  \y@tile_+\coloneqq \sup \bigl( Y^\prime_{[i, j, k]} \bigr)
  = \biggl\lfloor \frac{j}{2} \biggr\rfloor \, s + \biggl\lfloor \frac{j + 1}{2} \biggr\rfloor \, l \\
  &\z@tile_- \coloneqq \inf \bigl( Z^\prime_{[i, j, k]} \bigr)
  = \biggl\lfloor \frac{k + 1}{2} \biggr\rfloor \, s + \biggl\lfloor \frac{k}{2} \biggr\rfloor  \, l, \quad
  \z@tile_+\coloneqq \sup \bigl( Z^\prime_{[i, j, k]} \bigr)
  = \biggl\lfloor \frac{k + 2}{2} \biggr\rfloor \, s + \biggl\lfloor \frac{k + 1}{2} \biggr\rfloor \, l
\end{align*}

とする.
このとき,
\(d_\lambda \ (\lambda \in 
\{ \x@tile_\pm, \y@tile_\pm, \z@tile_\pm \})\)
を
\begin{align*}
  d_{\x@tile_\pm} \coloneqq
  \left\{ \begin{alignedat}{3}
    && \pm (a - \x@tile_\pm) &\quad (\text{if} \  \pm (a - \x@tile_\pm) \ge 0) \\
    &&0 &\quad (\text{if} \  \pm (a - \x@tile_\pm) < 0)
  \end{alignedat}
  \right. \\
  d_{\y@tile_\pm} \coloneqq
  \left\{ \begin{alignedat}{3}
    && \pm (b - \y@tile_\pm) &\quad (\text{if} \  \pm (b - \y@tile_\pm) \ge 0) \\
    &&0 &\quad (\text{if} \  \pm (b - \y@tile_\pm) < 0)
  \end{alignedat}
  \right. \\
  d_{\z@tile_\pm} \coloneqq
  \left\{ \begin{alignedat}{3}
    && \pm (c - \z@tile_\pm) &\quad (\text{if} \  \pm (c - \z@tile_\pm) \ge 0) \\
    &&0 &\quad (\text{if} \  \pm (c - \z@tile_\pm) < 0)
  \end{alignedat}
  \right.
\end{align*}
とする(複合同順).

このとき,ビー玉とタイルの距離 \(d\) は,
\begin{align}
  d = \sqrt{\sum_{\lambda \in \{ \x@tile_\pm, \y@tile_\pm, \z@tile_\pm \}} d_\lambda^2}
  - r
\end{align}
である.
}
